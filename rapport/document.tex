\documentclass[a4paper,12pt]{report}
\usepackage[utf8]{inputenc}
\usepackage[T1]{fontenc}
\usepackage[english,main=french]{babel}
\usepackage[babel=true]{csquotes}
\usepackage{graphicx}
\usepackage[export]{adjustbox}
\usepackage{amsmath}
\usepackage{algorithm2e}
\usepackage{hyperref}
\usepackage[toc,page]{appendix}
\usepackage[backend=biber,style=alphabetic]{biblatex}


\graphicspath{ {../images/} }
\addbibresource{refs.bib}
\nocite{*}
\RestyleAlgo{ruled}
\renewcommand{\appendixpagename}{Annexes}
\renewcommand{\appendixtocname}{Annexes}
\hypersetup{
    colorlinks=true, %colorise les liens
    breaklinks=true, %permet le retour  la ligne dans les liens trop longs
    urlcolor=blue, %couleur des hyperliens
    linkcolor=black, %couleur des liens internes
    bookmarksopen=true, %si les signets Acrobat sont crs, les afficher compltement.
}


\title{Rapport de stage}
\author{Oscar Buon}
\date{\today}


\begin{document}


\begin{titlepage}
    \centering
    \begin{minipage}{0.45\textwidth}
        \includegraphics[width=\textwidth]{logo_ISIMA_INP.png}
    \end{minipage}\hfill
    \begin{minipage}{0.45\textwidth}
        \includegraphics[width=\textwidth]{logo_CERN.png}
    \end{minipage}

    \vfill

	{\Large
        Rapport d'élève ingénieur \par
        Stage de 2ème année \par
        Filière : Informatique des systèmes interactifs pour l’embarqué, la robotique et le virtuel \par
    }

	\vfill

	{\huge\bfseries Speeding up LHCb software through compilation optimization \par}

	\vfill

	{\Large Présenté par : Oscar Buon \par}

	\vfill

	Responsable Entreprise : Sébastien Ponce \par
	Responsable ISIMA : Mamadou Kanté

	\vfill

	\today
\end{titlepage}


\pagenumbering{Roman}
\chapter*{Remerciements}
    J'aimerais commencer par remercier Sébastien Ponce, mon tuteur de stage sans qui mon travail au CERN n'aurait pas été possible.

    Je remercie égallement M. Mamadou Kanté, tuteur ISIMA et responsable de la fillière.

    \bigskip
    Je tiens aussi à remercier Alexandre Boyer qui m'a permis à l'occasion du forum ingénieur de trouver un stage au CERN.

\tableofcontents

\listoffigures


\begin{abstract}

    \vfill

    Mots-clés : LHCb, Optimisation, Compilation avancée, CMake

\end{abstract}

\begin{otherlanguage}{english}
\begin{abstract}

    \vfill

    Key words : LHCb, Optimization, Advanced compilation, CMake
\end{abstract}
\end{otherlanguage}


\pagenumbering{arabic}
\chapter*{Introduction}


\chapter{Introduction}
    \section{Présentation du CERN}
    L'Organisation européenne pour la recherche nucléaire est le plus grabd laboratoire d'étude de la physique des particules.
    Le centre, qui a été fondé en 1954, se situe sur la frontière franco-suisse à quelques kilomètres de Genève.

    Une grande partie des recherches éffectuées au CERN utilisent différents accélérateurs departicules.
    Le principe est de faire atteindre à des particules des vitesses proches de celle de la lumière pour qu'elles acquièrent une grande énergie, puis de les faire se collisionner.
    La physique des particules prévoit l'apparition de nouvelles particules lors de ces collisions.
    Ainsi en détectant les collisions au seins de l'accélérateur et en comparant avec la théorie, on peut la valdier ou l'invalider.

    Il existe plusieurs accélérateurs et plusieurs expériences au CERN.
    Le plus grand accélérateur est le Grand collisionneur de hadrons ou Large Hadron Collider (LHC) qui a été mis en fonction en 2008 et qui a une circonférence d'environ 27 kilomètres.
    Il permet d'accélérer des protons à environ 7 Tev.
    Sur le LHC sont installés 4 principales expériences qui sont des collaborations internationales : ATLAS, CMS, ALICE et LHCb.
    Ces expériences prennent la forme de détecteurs. C'est à l'intérieur d'eux que se font les collisions dont les résidus sont détectés par plusieurs types de capteurs.

    La mise en oeuvre des expériences du CERN a depuis longtemps nécessité le développement de nouvelles technologies.
    Ainsi l'informatique et internet sont présent depuis longtemps afin de gérer la masse de données produites dans les détecteurs.
    Des programmes comportants plusieurs millions de lignes de codes sont exécutées aux centres de calcul sur les sites du CERN mais aussi partout autour du monde grâce au "Grid".
    Le CERN est aussi par exemple à l'origine du Web et a contribué au développement d'internet en Europe.

    \section{L'expérience LHCb}
    Le Large Hadron Collider beauty est l'une des 4 principales expériences installées sur le LHC.
    Elle vise l'étude du quark b afin de mesurer l'asymétrie entre matière et antimatière.
    La collaboration regroupe plus de 1200 personnes représentant plusieurs dizaines d'instituts.
    L'expérience se situe sur le point 8 du LHC.

    \section{Contexte}
    Les différents instruments du capteur produisent un flux de données conséquent d'environ 3 To.
    Une importante infrastructure informatique est donc nécessaire pour pouvoir les traiter.

    LHCb utilise une pile de programmes (dont certains sont partagés avec d'autres expériences) qui traitent les données issues du détecteur.
    On peut citer : Geant, Root, Gaudi, LHCb, Lbcom, Rec et Moore.
    Cette pile contient plusieurs millions de lignes de codes qui peuvent avoir plusieurs décénies.
    Son maintient est donc un enjeux majeur.



\chapter{Méthode}
    \section{Fusion des bibliothèques et modules dynamiques}
        \subsection{Présentation}
            La pile LHCb est découpée en centaines de petites parties.
            Ce découpage existe d'abord pour des raisons historiques car cela permettait de plus facilement collaborer sur le même projet avant l'arrivée de Git.
            Mais cela permet aussi de ne pas avoir à tout compiler et charger lorsqu'on a besoin que de travailler sur une partie en particulier.

            De ce fait la pile une fois compilée se compose de centaines de bibliothèques et de modules dynamiques.
            Cette architecture est certes plus simple à utiliser mais il pourrait être intéressant d'étudier si leur fusion en un seul grand exécutable permettrait une meilleur performance en production.

        \subsection{Graphe des dépendances}
            Déjà,

    \section{Guided optimization}
        \subsection{Profile-guided optimization}

        \subsection{AutoFDO}

        \subsection{Link time optimization}

\chapter{Résultats}


\chapter*{Conclusion}


\printbibliography


\begin{appendices}


\end{appendices}

\end{document}
